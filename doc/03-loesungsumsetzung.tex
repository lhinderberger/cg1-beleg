\section{Lösungsumsetzung}
\subsection{Vorüberlegungen}
Als Programmiersprache wird C++ gewählt aufgrund der Möglichkeit, dort komfortabel 
objektorientiert zu programmieren.

Als Build-System wird CMake gewählt, welches die Entwicklung unter Linux ermöglicht und 
gleichzeitig die Erzeugung von Visual Studio-Projektdateien erlaubt.

Die Entwicklung erfolgte von Grund auf neu, ohne Einbezug des Rahmenprogramms aus der 
Vorlesung. Entwickelt wurde unter Linux.

\subsection{Entwicklung}
Das Programm wurde inkrementell, ausgehend von einem leeren Bildschirm aus 
objektorientiert entwickelt.

Zunächst wurde eine Klasse entworfen, welche die drei Raumachsen darstellt. Diese wurde 
aus der finalen Version der Anwendung wieder entfernt, war aber gerade am Anfang, vor 
der Fertigstellung der Kameraklassen essentiell, um nicht die Orientierung zu 
verlieren.

Als nächstes wurde mit der Entwicklung einer interaktiven First-Person-Kamera 
fortgefahren. Die Kamera sollte das Bewegen in der Ebene sowie die vollständige Drehung 
um die Y-Achse (Yaw, ''Sich-Umdrehen'') und die eingeschränkte Drehung um die Z-Achse 
(Pitch, ''Hoch- und Runtersehen'') ermöglichen. (Vier Freiheitsgrade)

Dann wurden die geometrischen Grundkörper Quader und Zylinder, zunächst noch ohne 
Normalen und Texturkoordinaten, implementiert und ohne Beleuchtung frei im Raum 
dargestellt. Zudem wurde eine Model-Klasse entworfen, welche unter einem gemeinsamen 
Dach die Modellmatrix und Aufrufe für das Rendering sammelt.

Damit konnte als nächstes die ''Leuchtreklame'' aus diesen Grundkörpern und diversen 
Transformationen zusammengesetzt werden, welche ihrerseits wieder ein Modell darstellt.

Fortgefahren wurde mit der Entwicklung der Beleuchtungsberechnung. Hierbei wurde das 
Phong-Beleuchtungsmodell gewählt und schrittweise zunächst Ambient-, dann Diffuse- und 
schließlich Specular-Lighting zunächst noch global mit einer Lichtquelle implementiert.
Dazu mussten für die geometrischen Grundkörper die Normalen berechnet definiert werden.
Im Zuge dessen wurde noch einmal die Erzeugung des geometrischen Grundkörpers des 
Quaders verändert, welcher jetzt nur noch Dreiecke (GL\_TRIANGLE) und keine 
Dreiecksstreifen mehr enthält. Zudem werden Ober- und Unterseite des Quaders jetzt nur 
noch aus einem VAO heraus gerendet, welcher verschoben und gedreht wird.

Im Folgenden Schritt wurden Materialeigenschaften für alle Modelle definiert und das 
Rendern mit mehreren definier- und veränderbaren Punktlichtquellen ermöglicht. Damit 
ließ sich die Leuchtreklame in ihrer (im Wesentlichen) abschließenden Form 
verwirklichen. Um die Leuchteffekte deutlicher sichtbar zu machen, wurde noch ein Boden 
hinzugefügt (für welchen noch ein Quadrat als geometrische Grundform geschaffen werden 
musste) sowie aus Quadern Würfel in der Szene verteilt.

Würfel sowie Boden wurden im nächsten Schritt texturiert, was das Einfügen von 
Texturkoordinaten in allen Grundkörpern sowie die Einbindung von FreeImage nach sich 
zog.

Danach wurde ein Würfel sowie die Leuchtreklame animiert, wozu noch einmal die Logik
des Renderns geändert werden musste, welches jetzt korrekterweise über
glutPostRedisplay() und die glutIdle()-Funktion erfolgt.

Schlussendlich wurde im Feinschliff noch eine zweite Kamera hinzugefügt, die finalen 
Kamerapositionen und -Blickwinkel bestimmt, das Overlay über den Bildschirm gelegt und 
die Darstellung in mehreren gleichzeitigen Viewports mit unterschiedlichen 
Kameras und Projektionen aktiviert, wozu die Render-Funktion noch einmal in mehrere 
Unterfunktionen aufgeteilt werden musste.

\subsection{Test und Portierung}
Nach erfolgreichem Test unter Linux musste laut Aufgabenstellung noch die Portierung 
nach Windows erfolgen. Nach einem erfolgreichem Test an meinem heimischen Rechner unter 
Visual Studio 2017 erfolgte zuletzt vor Ort in der HTW das Erstellen einer Visual 
Studio 2013-Projektmappe und der erforderlichen Windows-Binaries.
