\section{Lösungsansatz}
\subsection{Szene}
Da die Aufgabenstellung weitestgehende Gestaltungsfreiheit bezüglich der konkret zu 
realisierenden Szene lässt, fällt die Wahl auf eine Szene, welche folgende Objekte 
enthält:

\begin{itemize}
\item Ein flacher, mit einer Grastextur belegter Boden (Rechteck)
\item Eine Skybox, mit einer Wolken-/Sternetextur belegt (simuliert den Himmel)
\item Ein Quader und ein 3-Prisma, welche ein stilisiertes Haus darstellen sollen.
Der Quader Gelb gefärbt, das Prisma mit einer Dachziegeltextur belegt.
\item Mehrere stilisierte Straßenlaternen, bestehend aus einen Zylinder und einem 
Quader, grau gefärbt.
\item Eine aus drei Quadern bestehende stilisierte Straße, Farben Dunkel- und Hellgrau.
\item Eine Menge von Zylindern und Quadern, welche ein stilisiertes Auto ergeben.
Bunt gefärbt.
\end{itemize}

Die Objekte sind in sinnvoller Art und Weise anzuordnen.

\subsection{Lichtquellen}
Im Tagmodus soll durch globale Beleuchtung die ganze Szene gut sichtbar sein. Im 
Nachtmodus wird die globale Beleuchtung auf ein Minimum reduziert und stattdessen helle 
Spotlights in den Straßenlaternen aktiviert, welche auf dem vorbeifahrenden Auto 
Lichteffekte produzieren. Außerdem sollen vorne und hinten am Auto durch punktförmige
Lichtquellen die Vorder- und Rücklichter simuliert werden.

\subsection{Animation}
Das Auto fährt in einer sinusförmig an- und abschwellenden Bewegung auf der geraden 
Straße vor- und wieder rückwärts an den Straßenlaternen vorbei.

\subsection{Viewports und Kameras}
Die Szene soll gleichzeitig durch zwei Kameras in jeweils zwei Projektionen 
(Orthogonale Projektion und Perspektivische Projektion) beobachtbar sein.

Dabei sollen die beiden linken Drittel des Fensters durch den Haupt-Viewport ausgefüllt 
werden und auf dem rechten Drittel sollen die drei weiteren Viewports vertikal 
übereinander angezeigt werden. Ein Wechsel zwischen den Kameras ist jederzeit möglich 
(siehe Unten).

\subsection{Interaktion}
Die verfügbaren Interaktionen sollen dem Nutzer durch Bildschirm-Overlay angezeigt 
werden. Das Programm wird somit selbsterklärend.

Folgende Eingriffe durch den Benutzer werden ermöglicht:

\begin{itemize}
\item Ein-/Ausblenden des Overlays (Taste O)
\item Umschalten der derzeit aktiven Kamera (Tasten 1-4)
Dabei wird die jeweils aktive Kamera in den Haupt-Viewport geholt und die anderen 
Kameras werden in die Neben-Viewports verschoben.
\item Bewegung der derzeit aktiven Kamera in der Ebene (Pfeiltasten oder W,A,S,D)
\item Bewegungsgeschwindigkeit erhöhen (Shift gedrückt halten)
\item Drehung und Neigung der Kamera (Bewegung der Maus)
\item Erhöhen-/Senken der Animationsgeschwindigkeit (Tasten +/-)
\item Umschalten zwischen Tag- und Nachtmodus (Taste N)
\item Umschalten zwischen gefülltem, texturierten, und Wireframe-Modus (Taste M)
\item Ein-/Ausblenden eines Koordinatensystems (Taste K)
\item Nur in Perspektivischer Projektion: Erhöhen-/Senken des Blickwinkels (Tasten Bild 
Auf / Ab)
\end{itemize}

