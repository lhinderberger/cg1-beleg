\section{Lösungsansatz}
\subsection{Szene}
Da die Aufgabenstellung weitestgehende Gestaltungsfreiheit bezüglich der konkret zu
realisierenden Szene lässt, fällt die Wahl auf eine Szene, welche folgende Objekte
enthält:

\begin{itemize}
\item Eine Skybox, mit einer Wolken-/Sternetextur belegt (simuliert den Himmel)
\item Ein Quader, welcher ein Hochhaus darstellen soll und mit einer entsprechenden
Textur belegt ist.
\item Ein weiterer Quader mit zahlreichen, darauf befestigten Halbkugeln, welcher
eine Glühbirnen-Leuchtreklame mit dem Text "OPEN GL" darstellen soll.
Die Leuchtreklame ist an der Fassade des Hochhauses angebracht.
\end{itemize}

\subsection{Lichtquellen}
Im Tagmodus soll durch globale Beleuchtung die ganze Szene gut sichtbar sein. Im
Nachtmodus wird die globale Beleuchtung auf ein Minimum reduziert.

Die Glühbirnen der Leuchtreklame leuchten farbig.

\subsection{Animation}
Die Wörter "OPEN" und "GL" leuchten in verschiedenen, interaktiv wählbaren
Blinkmustern auf.

\subsection{Viewports und Kameras}
Die Szene soll gleichzeitig durch zwei Kameras in jeweils zwei Projektionen
(Orthogonale Projektion und Perspektivische Projektion) beobachtbar sein.

Dabei sollen die beiden linken Drittel des Fensters durch den Haupt-Viewport ausgefüllt
werden und auf dem rechten Drittel sollen die drei weiteren Viewports vertikal
übereinander angezeigt werden. Ein Wechsel zwischen den Kameras ist jederzeit möglich
(siehe Unten).

\subsection{Interaktion}
Die verfügbaren Interaktionen sollen dem Nutzer durch Bildschirm-Overlay angezeigt
werden. Das Programm wird somit selbsterklärend.

Folgende Eingriffe durch den Benutzer werden ermöglicht:

\begin{itemize}
\item Ein-/Ausblenden des Overlays (Taste O)
\item Umschalten der im Haupt-Viewport aktiven Kamera (Tasten K)
\item Umschalten der im Haupt-Viewport aktiven Projektion (Taste P)
\item Bewegung der derzeit aktiven Kamera in der Ebene (Pfeiltasten oder W,A,S,D)
\item Bewegungsgeschwindigkeit erhöhen (Shift gedrückt halten)
\item Drehung und Neigung (Pitch/Yaw) der Kamera (Bewegung der Maus)
\item Umschalten des Blinkmusters (Taste B)
\item Erhöhen-/Senken der Animationsgeschwindigkeit (Tasten +/-)
\item Umschalten zwischen Tag- und Nachtmodus (Taste N)
\item Umschalten zwischen gefülltem, texturierten, und Wireframe-Modus (Taste M)
\item Nur in Perspektivischer Projektion: Erhöhen-/Senken des Blickwinkels (Tasten Bild
Auf / Ab)
\item Verlassen der Anwendung (Taste Q)
\end{itemize}
