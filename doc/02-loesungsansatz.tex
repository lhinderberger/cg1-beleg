\section{Lösungsansatz}
\subsection{Szene}
Da die Aufgabenstellung weitestgehende Gestaltungsfreiheit bezüglich der konkret zu
realisierenden Szene lässt, fällt die Wahl auf eine Szene, welche folgende Objekte
enthält:

\begin{itemize}
\item Ein Quader mit zahlreichen, darauf befestigten kleinen Zylindern,
welche eine Glühbirnen-Leuchtreklame mit dem Text "Open GL" darstellen sollen.
\item Mehrere im Raum verteilte Würfel, welche mit einer Spielwürfel-Textur belegt sind.
\item Ein großes Rechteck, welches den Boden darstellt und mit einer Holztextur
belegt ist.
\end{itemize}

\subsection{Lichtquellen}
Durch globale direktionale Beleuchtung soll die ganze Szene gut sichtbar sein.
Die Glühbirnen der Leuchtreklame leuchten farbig.

\subsection{Animation}
Die Buchstaben der Leuchtreklame leuchten in verschiedenen, interaktiv wählbaren
Blinkmustern auf.
Einer der Würfel schwebt in der Höhe und rotiert, sodass darauf gut Lichteffekte sichtbar
werden.

\subsection{Viewports und Kameras}
Die Szene soll gleichzeitig durch zwei Kameras in jeweils zwei Projektionen
(Orthogonale Projektion und Perspektivische Projektion) beobachtbar sein.

Dabei sollen die beiden linken Drittel des Fensters durch den Haupt-Viewport ausgefüllt
werden und auf dem rechten Drittel sollen die vier Kameraansichten vertikal
übereinander angezeigt werden. Ein Wechsel zwischen den Kameras ist jederzeit möglich
(siehe Unten).

\subsection{Interaktion}
Die verfügbaren Interaktionen sollen dem Nutzer durch Bildschirm-Overlay angezeigt
werden. Das Programm wird somit selbsterklärend.

Folgende Eingriffe durch den Benutzer werden ermöglicht:

\begin{itemize}
\item Ein-/Ausblenden des Overlays (Taste O)
\item Umschalten der im Haupt-Viewport aktiven Kamera (Tasten 1-4)
\item Bewegung der derzeit aktiven Kamera in der Ebene (Pfeiltasten oder W,A,S,D)
\item Drehung und Neigung (Pitch/Yaw) der Kamera (Bewegung der Maus)
\item Umschalten des Blinkmusters (Taste B)
\item Umschalten zwischen gefülltem, texturierten, und Wireframe-Modus (Taste M)
\item Verlassen der Anwendung (Taste Q)
\end{itemize}
