\section{Aufgabenbeschreibung}
\subsection{Anforderungen}
\begin{itemize}
\item Erzeugen einer Szene mit mehreren unterschiedlichen dreidimensionalen 
geometrischen Objekten.\footnote{Unter geometrischen Objekten werden hierbei geometrische Grundkörper verstanden, die 
als solche ersichtlich sein sollen, jedoch beliebig angeordnet werden können um auch 
komplexere Objekte anzudeuten.}

\item Die dreidimensionalen Objekte sollen zum Teil gefärbt, zum Teil texturiert sein.

\item Die Szene ist zeitlich zu animieren.

\item Der Nutzer soll in die Szene eingreifen können (interaktiv), z.B. sich in
der Szene bewegen, die Animation beeinflussen usw.

\item Die Szene soll durch mehrere Lichtquellen beleuchtet werden.

\item Auf den dargestellten Objekten sollen unterschiedliche Beleuchtungseffekte
sichtbar werden.

\item Es sollen unterschiedliche Ansichten und Projektionen (z.B. Orthogonale
Projektion, Perspektivische Projektion) gleichzeitig in mehreren Viewports
dargestellt werden.

\item Als Programmiersprache wird C/C++ vorausgesetzt.

\item Als Grafikschnittstelle wird OpenGL vorausgesetzt.

\item Vertex- und Fragmentshader müssen eingesetzt werden.

\item Der Programmteil muss in ausführbarer Form bereitgestellt werden.

\item Als Abgabeformat für den Quelltext wird eine Microsoft Visual-Studio-Projektdatei gefordert.

\item Die Dokumentation muss im PDF-Format eingereicht werden.

\item Die Dokumentation soll etwa 10 Seiten umfassen und Deckblatt, Gliederung, 
Aufgabenbeschreibung, Lösungsansatz, Installations- und Bedienungsanleitung, einige 
Bildschirm-Snapshots, Probleme, Ergebnisse, Literatur- und Quellenverzeichnis
enthalten.
\end{itemize}
