\section{Analyse}
\subsection{Aufgabenstellung}
Die Aufgabenstellung wird nun im Wortlaut wiedergegeben:

Schreiben Sie ein Programm in C/C++, das unter Verwendung von OpenGL, Vertex- und 
Fragment-Shadern folgende Aufgaben realisiert.

\textbf{Aufgabe 1:}

Geometrische Objekte: Erzeugen Sie eine interaktive zeitlich animierte Szene 
mit mehreren unterschiedlichen farblichen und texturierten dreidimensionalen 
geometrischen Objekten.

\textbf{Aufgabe 2:}

Beleuchten Sie die Szene mit mehreren Lichtquellen so, dass auf den Objekten 
unterschiedliche Beleuchtungseffekte sichtbar werden.

\textbf{Aufgabe 3:}

Ansicht: Stellen Sie die Szene gleichzeitig in verschiedenen Ansichten und 
Projektionen in mehreren Viewports des Anzeigefensters dar.

\textbf{Aufgabe 4:}

Dokumentation: Stellen Sie das programm in ausführbarer Form mit Quelltext 
als Visual-Studio-C++-Projekt bereit. Fertigen Sie eine Systemdokumentation in Form 
eines pdf-Dokumentes von etwa 10 Seiten an, die Deckblatt, Gliederung, 
Aufgabenbeschreibung, Lösungsansatz, Installations- und Bedienungsanleitung, einige 
Bildschirm-Snapshots, Probleme, Ergebnisse, Literatur- und Quellenverzeichnis enthält.


\subsection{Anforderungen}
Aus oben wiedergegebener Aufgabenstellung werden folgende Anforderungen ersichtlich, 
die im Folgenden konkretisiert werden:

\begin{itemize}
\item Erzeugen einer Szene mit mehreren unterschiedlichen dreidimensionalen 
geometrischen Objekten.\footnote{Unter geometrischen Objekten werden hierbei geometrische Grundkörper verstanden, die 
als solche ersichtlich sein sollen, jedoch beliebig angeordnet werden können um auch 
komplexere Objekte anzudeuten.}

\item Die dreidimensionalen Objekte sollen zum Teil gefärbt, zum Teil texturiert sein.

\item Die Szene ist zeitlich zu animieren. Die dargestellen Objekte sollen sich also 
selbständig bewegen können.

\item Der Nutzer soll in die Szene eingreifen können (interaktiv), z.B. sich in der 
Szene bewegen, die Geschwindigkeit der Animation beeinflussen usw.

\item Die Szene soll durch mehrere Lichtquellen beleuchtet werden. Dies soll in einer 
Weise geschehen, sodass auf den dargestellten Objekten Beleuchtungseffekte sichtbar 
werden.

\item Es sollen unterschiedliche Ansichten (gemeinhin Kameras) und Projektionen (z.B. 
Orthogonale Projektion, Perspektivische Projektion) gleichzeitig in mehreren Viewports
dargestellt werden.

\item Als Programmiersprache wird C oder C++ vorausgesetzt.

\item Als Grafikschnittstelle wird OpenGL vorausgesetzt.

\item Vertex- und Fragmentshader müssen eingesetzt werden.

\item Nach klärender Nachfrage: Als Schnittstelle zum Betriebssystem wird zwingend 
FreeGLUT vorausgesetzt.

\item Als Abgabeformat für den Programmteil wird zwingend eine Microsoft 
Visual-Studio-Projektdatei gefordert.

\item Als Abgabeformat für die Dokumentation wird zwingend PDF gefordert.

\item Die Dokumentation soll etwa 10 Seiten umfassen und wie oben beschriebene Punkte 
beinhalten.
\end{itemize}
